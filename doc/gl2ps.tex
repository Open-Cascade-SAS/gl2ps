%
%  GL2PS, an OpenGL to Postscript Printing Library
%  Copyright (C) 1999-2001  Christophe Geuzaine 
%
%  $Id: gl2ps.tex,v 1.11 2001-06-12 08:48:03 geuzaine Exp $
% 
%  E-mail: Christophe.Geuzaine@advalvas.be
%  URL: http://www.geuz.org/gl2ps/
% 
%  This library is free software; you can redistribute it and/or
%  modify it under the terms of the GNU Library General Public
%  License as published by the Free Software Foundation; either
%  version 2 of the License, or (at your option) any later version.
% 
%  This library is distributed in the hope that it will be useful,
%  but WITHOUT ANY WARRANTY; without even the implied warranty of
%  MERCHANTABILITY or FITNESS FOR A PARTICULAR PURPOSE.  See the GNU
%  Library General Public License for more details.
% 
%  You should have received a copy of the GNU Library General Public
%  License along with this library; if not, write to the Free
%  Software Foundation, Inc., 675 Mass Ave, Cambridge, MA 02139, USA.
% 

\documentclass[10pt]{article}

\pagestyle{headings}
\setcounter{tocdepth}{2}

\ifx\pdfoutput\undefined
  \usepackage{url}
\else
  \usepackage[colorlinks=true,urlcolor=blue]{hyperref}
  \pdfinfo {
    /Title (GL2PS, an OpenGL to Postscript Printing Library)
    /Author (Christophe Geuzaine)
  } 
\fi
\newcommand\dd{\begingroup\Url}
\newcommand\email{\dd}
%%tth:\def\dd#1{\textmd{\texttt{#1}}}
%%tth:\def\url#1{\href{#1}{#1}}
%%tth:\def\email#1{\href{mailto:#1}{#1}}

\begin{document}

%%tth:\special{html:<BODY TEXT="#000000" BGCOLOR="#ffffff">}

\title{GL2PS, an OpenGL to Postscript Printing Library}
\author{Christophe Geuzaine}
\date{Version 0.4, 12 June 2001}

\maketitle

%%tth:\section*{Download}
%%tth:The current archive of the package is available
%%tth:\href{http://www.geuz.org/gl2ps/gl2ps-0.4.tar.gz}{here}.

\tableofcontents

\section{Introduction}

GL2PS is a library for creating postscript output from any OpenGL
application. Though it was primarily designed for three-dimensional geometry,
mesh and postprocessing visualization, it may be useful everytime high quality
vector output is desired. The main difference between GL2PS and other similar
libraries is the use of sorting algorithms capable of handling intersecting and
stretched polygons, as well as non manifold objects.

The library, written in C, is released under GNU Library General Public
License (see \url{http://www.gnu.org/} for more details), and is available
at \url{http://www.geuz.org/gl2ps/}. Any corrections, questions or
suggestions should be e-mailed to \email{Christophe.Geuzaine@advalvas.be}.

The interface consists of five functions, all begining with the prefix
\dd{gl2ps}. All the data structures and the symbolic constants peculiar to
GL2PS begin with \dd{GL2PS}.


\section{Usage}

\subsection{\texttt{gl2psBeginPage} and \texttt{gl2psEndPage}}

\subsubsection{Specification}

\begin{verbatim}
void gl2psBeginPage( char *title, char *producer, GLint sort,
                     GLint options, GLint colormode,
                     GLint colorsize, GL2PSrgba *colortable, 
                     GLint buffersize, FILE *stream )
\end{verbatim}

\begin{verbatim}
void gl2psEndPage( void )
\end{verbatim}

\subsubsection{Parameters}

\begin{description}
\item[\dd{title}] Specifies the plot title. For Postscript output, this string
is placed in the \texttt{\%\%Title} field.

\item[\dd{producer}] Specifies the plot producer. For Postscript output, this
string is placed in the \texttt{\%\%For} field.

\item[\dd{sort}] Specifies the sorting algorithm, chosen among:
\dd{GL2PS_NO_SORT}, \dd{GL2PS_SIMPLE_SORT}, \dd{GL2PS_BSP_SORT}.

\item[\dd{options}] Sets global plot options, chosen among:
\dd{GL2PS_NONE}, \dd{GL2PS_DRAW_BACKGROUND}, \dd{GL2PS_SIMPLE_LINE_OFFSET},
\dd{GL2PS_SILENT}, \dd{GL2PS_BEST_ROOT}. Multiple options are combined with the
bitwise inclusive OR symbol,~\dd{|}.

\item[\dd{colormode}] Specifies the color mode: \dd{GL_RGBA} or
\dd{GL_COLOR_INDEX}.

\item[\dd{colorsize}] Specifies the size of the colormap if \dd{colormode} is
\dd{GL_COLOR_INDEX}.

\item[\dd{colortable}] Contains the colormap if \dd{colormode} is
\dd{GL_COLOR_INDEX}. This colormap must contain \dd{colorsize} elements of type
\dd{GL2PSrgba}.

\item[\dd{buffersize}] Specifies the size of the feedback buffer.

\item[\dd{stream}] Specifies the stream to which data is printed.
\end{description}

\subsubsection{Description}

\dd{gl2psBeginPage} and \dd{gl2psEndPage} delimit the OpenGL commands that will
be caught in the feedback buffer and output to \dd{stream}. The parameters
given to \dd{gl2psBeginPage} determine the way primitives are handled:

\begin{description}
\item[\dd{GL2PS_NO_SORT}] The primitives are not sorted, and are output in
\dd{stream} in the order they appear in the feedback buffer.

\item[\dd{GL2PS_SIMPLE_SORT}] The primitives are sorted according to their
barycenter. This can be sufficient for simple scenes.

\item[\dd{GL2PS_BSP_SORT}] The primitives are inserted in a BSP tree. The tree
is traversed back to front in a painter-like algorithm.

\item[\dd{GL2PS_DRAW_BACKGROUND}] The background frame is drawn.

\item[\dd{GL2PS_SIMPLE_LINE_OFFSET}] Adds a small offset in the z-buffer to all
lines. This is a simplified version of the \dd{GL2PS_POLYGON_OFFSET_FILL}
functionality (cf. section~\ref{sec:gl2psEnable}), putting all lines of the
rendered image slightly in front of their actual position. This thus performs a
simple anti-aliasing solution, e.g. for finite element like meshes.

\item[\dd{GL2PS_SILENT}] Suppresses all messages written by GL2PS on the error
stream.

\item[\dd{GL2PS_BEST_ROOT}] Try to optimize the BSP tree by choosing as root
primitives those leading to the minimum number of splits. This is (really) not
efficient yet.

%\item[\dd{GL2PS_OCCLUSION_CULL}] Removes (well, tries to remove) all hidden
%polygons from the output, thus substancially reducing the size of the output
%file. This is not functionnal yet.
\end{description}


\subsection{\texttt{gl2psText}}

\subsubsection{Specification}

\begin{verbatim}
void gl2psText( char *string, char *fontname, GLint fontsize )
\end{verbatim}

\subsubsection{Parameters}

\begin{description}
\item[\dd{string}] Specifies the text string to print.

\item[\dd{fontname}] Specifies the name of a valid postscript font (for example
\dd{"Times"} or \dd{"HelveticaBoldItalic"}).

\item[\dd{fontsize}] Specifies the size of the font.
\end{description}

\subsubsection{Description}

\dd{gl2psText} permits to include text in the postscript output in a very
simple way. The text is inserted at the current raster position (set by one
of the \dd{glRasterPos} OpenGL commands). Beware that text will be sorted
according to the position of the leftmost element of the string only.

\subsection{\texttt{gl2psEnable} and \texttt{gl2psDisable}}
\label{sec:gl2psEnable}

\subsubsection{Specification}

\begin{verbatim}
void gl2psEnable( GLint mode )
\end{verbatim}

\begin{verbatim}
void gl2psDisable( GLint mode )
\end{verbatim}

\subsubsection{Parameters}

\begin{description}
\item[\dd{mode}] Specifies the mode to enable, chosen between
\dd{GL2PS_POLYGON_OFFSET_FILL}, \dd{GL2PS_POLYGON_BOUNDARY},
\dd{GL2PS_LINE_STIPPLE}.
\end{description}

\subsubsection{Description}

\dd{gl2psEnable} and \dd{gl2psDisable} delimit OpenGL commands to which a local
mode is applied. These modes are:


\begin{description}
\item[\dd{GL2PS_POLYGON_OFFSET_FILL}] Tries to emulate the
\dd{GL_POLYGON_OFFSET_FILL} functionnality. The value of the offset is taken as
the current value of the corresponding OpenGL offset (set with
\dd{glPolygonOffset}). Not fully functionnal yet.

\item[\dd{GL2PS_POLYGON_BOUNDARY}] Not implemented yet.

\item[\dd{GL2PS_LINE_STIPPLE}] Tries to emulate the \dd{GL_LINE_STIPPLE}
functionnality.
\end{description}

\subsection{\texttt{gl2psPointSize} and \texttt{gl2psLineWidth}}
\label{sec:gl2psPointSize}

\subsubsection{Specification}

\begin{verbatim}
void gl2psPointSize( GLfloat value )
\end{verbatim}

\begin{verbatim}
void gl2psLineWidth( GLfloat value )
\end{verbatim}

\subsubsection{Parameters}

\begin{description}
\item[\dd{value}] Specifies the size or the width of points and lines.
\end{description}

\subsubsection{Description}

\dd{gl2psPointSize} and \dd{gl2psLineSize} emulate the standard
\dd{glPointSize} and the \dd{glLineWidth} functions. They are necessary
since the point sizes and line widths are not saved in the OpenGL feedback
buffer.

\section{Example}

Here is a typical calling sequence to produce BSP sorted postscript output in
the file \dd{"MyFile"}, with all lines slightly shifted front in the
z-buffer. The \dd{draw()} function contains all OpenGL instructions.

\begin{verbatim}
fp = fopen("MyFile", "w");
buffsize = 0;
state = GL2PS_OVERFLOW;

while( state == GL2PS_OVERFLOW ){ 
  buffsize += 1024*1024;
  gl2psBeginPage ( "MyTitle", "MySoftware", GL2PS_BSP_SORT,
                   GL2PS_SIMPLE_LINE_OFFSET | GL2PS_SILENT, 
                   GL_RGBA, 0, NULL, buffsize, fp );
  draw(); 
  state = gl2psEndPage();
}

fclose(fp);
\end{verbatim}

\noindent To output the text \dd{"MyText"} at the current raster position, the
\dd{draw()} function should contain something like:

\begin{verbatim}
gl2psText("MyText", "Courier", 12);
\end{verbatim}

%\section{Implementation Notes}
%
%OpenGL feedback buffer, BSP trees


\section{Contributors}

Michael Sweet (\email{mike@easysw.com}) for the original implementation of the
feedback buffer parser; Marc Um{\'e} (\email{marc.ume@digitalgraphics.be}) for
the original list code;
%Bruce Naylor (\email{naylor@comp-graphics.com}) for some BSP tree and occlusion
%culling hints;
Jean-Fran\c{c}ois Remacle (\email{remacle@scorec.rpi.edu}) for plane equation
fixes;
Bart Kaptein (\email{B.L.Kaptein@lumc.nl}) for memory leaks fixes.

Projects similar to \dd{GL2PS} include: Michael Sweet's GLP library
(\url{http://dns.easysw.com/~mike/opengl/index.html}); Mark J. Kilgard's
rendereps (\url{http://reality.sgi.com/opengl/tips/Feedback.html}); the
GLpr library from CEI international (\url{http://www.ceintl.com/}).

\section{Versions}

\begin{description}
\item[0.1 (Feb 12, 2000)] First distributed version.
\item[0.2 (Feb 20, 2000)] Added \dd{GL2PS_POLYGON_BOUNDARY} and
\dd{GL2PS_BEST_ROOT}. Changed arguments of \dd{gl2psBeginPage} and
\dd{gl2psText}. Corrected some memory allocation stuff. First version of
this user's guide.
\item[0.21 (Mar 16, 2000)] Initialization fixes.
\item[0.3 (Jul 29, 2000)] Code cleaning. Added \dd{GL2PS_LINE_STIPPLE}.
\item[0.31 (Aug 14, 2000)] Better handling of erroneous primitives.
\item[0.32 (May 23, 2001)] Fixed memory leaks.
\item[0.4 (Jun 12, 2001)] Added \dd{gl2psPointSize} and
\dd{gl2psLineWidth}. Some code cleaning to allow easier generation of vector
file formats other than postscript.
% Occlusion culling.
\end{description}

\end{document}

